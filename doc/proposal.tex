\documentclass[12pt]{article}
\usepackage[margin=0.5in]{geometry}
\usepackage{titling}
\usepackage[compact]{titlesec}

\setlength{\droptitle}{-4em}
\addtolength{\droptitle}{-4pt}

\title{Preliminary Project Proposal}
\author{ Max Thrun \\ Samir Silbak }

\begin{document}
\maketitle

%\section{Project Title}
\section*{Project Title}
Sega Game Gear on a Chip (SGGoC)

\section*{Problem Statement}
Since the introduction of gaming consoles there has been
much effort put into emulating these systems in software on computers. 
There has not, however, been as much effort put into the physical reimplementation of their hardware.
A brief survey shows only a handful of projects that have reimplemented game consoles such as the Nintendo NES, 
the TurboGrafx-16, and the Gameboy on FPGAs. The Sega Game Gear, however, seem to be untouched as we were not
able to find an open source FPGA reimplementation. We believe reimplementing these classic game consoles in hardware
is an important part of understanding, documenting, and educating people on their design and functionality. 
For our project we plan to fully recreate the Sega Game Gear hardware on a FPGA in the hope that our project
can be used as an educational tool in the area of digital system design.

\section*{Potential Approaches}
The Game Gear hardware can be easily broken down into submodules. Major components include the Zilog Z80 CPU, 
the Video Display Processor (VDP) which is a modified Texas Instruments TMS9918, the Sega IO controller, and the
game cartridge memory mappers. The implementation of the Zilog Z80 is outside the scope of this project and as such
we will be using the popular open source TV80 CPU. A memory management unit will be developed to coordinate the 
addressing of system RAM and the cartridge ROM. The cartridge ROM will initially be preloaded on a flash memory chip on
our development board. If time allows a proper bootloader may be developed to allow game ROMs to be selected off a
SD card. We plan on implementing the submodules in order of priority: TV80 CPU, MMU, Sega Cartridge Memory Mapper, VDP,
Sega IO Controller, Audio (YM2413).

\section*{Final Implementation Description}
Our final implementation will be a fully functioning Sega Game Gear running on an Altera DE-1 FPGA
development board. Video output will be via VGA to a computer monitor and input will be through some type of retro
gaming controller, such as the Sega Genesis controllers. Any Sega Game Gear ROM which uses the Sega mapper 
(we do not plan on implementing less common mappers) will be playable.

\section*{Team Participants}
{\bf Max Thrun} - FPGAs / computer architecture / programming  
\\
{\bf Samir Silbak} - Linux / embedded systems / software development

\section*{Advisor}
Carla Purdy (carla.purdy@uc.edu)

\end{document}
